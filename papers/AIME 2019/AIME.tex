\documentclass[runningheads,a4paper]{llncs}

\usepackage[cmex10]{amsmath}
\usepackage{graphicx}
\usepackage{booktabs}
\usepackage{tabularx}
\usepackage{rotating} 
\usepackage{color}
\usepackage{subfig}
\usepackage[unicode,
					  colorlinks,
						linkcolor={red},
						citecolor={blue},
						urlcolor={blue},
            pdfauthor={Maciej Piernik, Joanna Solomiewicz, Arkadiusz Jachnik},
            pdftitle={State--changing sequential patterns},
            pdfsubject={Sequential pattern mining},
            pdfkeywords={sequential data, frequent patterns, uplift modeling}]{hyperref}

\newcolumntype{R}{>{\raggedleft\arraybackslash}X}
\newcommand{\argmax}[1]{\underset{#1}{\operatorname{argmax}}\;}
\newcommand{\matr}[1]{\mathbf{#1}}

\begin{document}

\mainmatter

\title{Assessing the effectiveness of sequences of treatments using sequential patterns}

\titlerunning{State--changing sequential patterns}

\author{Maciej Piernik \and Joanna Solomiewicz \and Arkadiusz Jachnik}

\authorrunning{Maciej Piernik \and Joanna Solomiewicz \and Arkadiusz Jachnik}

\institute{Institute of Computing Science, Poznan University of Technology\\
ul. Piotrowo 2, 60--965 Poznan, Poland\\
\email{maciej.piernik@cs.put.poznan.pl}}

\maketitle

\begin{abstract}
In this paper, we tackle the issue of assessing the effectiveness of sequences of treatments by introducing the concept of state--changing sequential patterns.
Our proposal aims at identifying sequential patterns in an environment where certain actions are taken for patients (medical procedures, administration of pharmaceuticals, etc.) while simultaneously measuring some indicator of their health (e.g., blood pressure).
We propose to combine the information about the events with the information about the states of the patients targeted by these events when mining for sequential patterns.
To be able to properly interpret the changes in states as outcomes of sequences of events, we rely on the concept of a control group, known from clinical trials.
We illustrate the usefulness of our proposal with an experiment.
\keywords{sequential data, frequent patterns, uplift modeling}
\end{abstract}

\section{Introduction}
\label{sec:introduction}

Sequential patterns are an extension of frequent patterns (or frequent itemsets, known from association rule mining) to sequential data.
They find many applications in domains such as customer transaction analysis, web mining, software bug analysis, chemical and biological analysis~\cite{Aggarwal:2014}.
Just like with traditional frequent patterns, there are many versions of sequential patterns, depending on the structure of the sequences.
For instance, each event in a sequence can be either a single item or an itemset; precedence of elements can be implied solely by their order or explicitly by time; time can be used to further narrow the problem by constraining time gaps between elements; each event in a sequence can be described by a symbol, a number, or a set of features, etc.~\cite{Dong:2009}.
On top of that, in scenarios such as classification or regression, target attribute can be added to each element in each sequence.
This results in a setting where a dataset contains sequences of pairs $\langle\texttt{event}, \texttt{target}\rangle$.
In many real-world scenarios, however, such a setting is impossible to achieve, as the value of the target attribute may be provided with a delay or even completely asynchronously from the analyzed events.
Consider a sequence of treatments prescribed to a given patient for a certain disease measured by some indicator (e.g., blood pressure).
After a series of events (e.g., administered pharmaceuticals, medical procedures, dietary regulations) the indicator may either improve, worsen, or stay unchanged.
However, this result does not necessarily coincide with any of the events nor need it be a result of one, all, or any of the events.
This scenario is universal when modeling people's behavior, opinion, or --- more generally speaking --- \textit{state}.
As illustrated by the examples above, this problem is no longer described by a single sequence of events (like in classical sequential pattern mining), but rather by two connected sequences --- one with the events and the other with target values.
To the best of our knowledge, processing of sequential data of such composition has not yet been considered and is the focus of this research.

In this work--in--progress paper, we introduce the concept of state--changing sequential patterns along with several possible options of solving it.
Inspired by clinical trials, these proposed solutions rely on the notion of a control group.
We showcase the applicability of the presented concept in practical situations by performing an experiment or real--world dataset.

Unlike in regular clinical trials where we try to nullify the impact of all other factors, state--changing sequential patterns allow for discovering potentially hidden dependencies between medical events from completely independent historical data where different events may appear in different histories.

\section{Related Work}
\label{sec:related}
Sequential pattern mining has first been introduced by Agrawal and Srikant~\cite{Agrawal:1995} through a market basket analysis model.
Since mining of such patterns is very costly, many optimisation algorithms being created to improve sequential pattern mining.
Giannotti et al.~\cite{Giannotti:2006} propose an annotation solution to a problem of distinction between patterns with the same sequence but different transition times.
Gebser et al.~\cite{Gebser:2016} propose to use knowledge-based sequence mining which takes into account expert knowledge in order to extract fewer patterns but of greater relevance.

Associating data with additional information not only can help in pattern distinction or evaluation of relevance but also in classifying it into categories.
This was suggested by Pinto et al.~\cite{Pinto:2001}.
Their algorithm focuses on multi-dimensional data and describes how certain patterns might apply to certain categories of data therefore the use of classification of patterns for their distinction.
Multi-dimensional data has also been examined by Plantevit et al.~\cite{Plantevit:2010}.
The framework that has been presented by them concentrates on relevant frequent sequences in multi-dimensional and multi-level data.
It is a solution to mining relevant patterns in data of various dimensions, but there are other proposals for standard sequential data.
One of such papers~\cite{Fowkes:2016} proposes an algorithm for mining the most relevant sequential patterns and also provides a ranking according to their interestingness.
Another paper~\cite{Li:2015} about mining interesting sequential patterns uses leverage (difference between observed and expected frequencies of a pattern) as a measure of interest.

A solution to mining patterns with a user-centric approach has been described by Guidotti et al.~\cite{Guidotti:2017}.
In their market basket prediction model the focus is on single user’s history by using four characteristics: co-occurency (items often bought together), sequentiality (set of items often bought after another one), periodicity (sequential purchases in specific periods), recurrence (frequency of sequential purchases in a given period).

All papers mentioned in this section refer to sequential patterns, however, none of them studies patterns with an impact on certain objects' state.
To the best of our knowledge, such a problem has not yet been considered.

\section{State--changing sequential patterns}
\label{sec:main}
Assume we have a history of medical events (performed procedures, administered pharmaceuticals, dietary regulations, etc.) of a given patient.
Additionally, between these events the health of the patient was being monitored and recorded in a form of some indicators (e.g., blood test results).
Now, given a database of such records for many patients, we can look for patterns of events which increase the chances of improving the patients' health.
Typically, one would analyze each medical event in isolation from others to assess it's sole impact on patient's health, which is an appropriate approach when performing clinical trials.
However, given historical data of the above--described composition, we can look for patterns of different events appearing in a certain order, i.e., sequential patterns.

Given the above, the problem of state--changing sequential patterns can be formulated as follows.
Is it possible to find a sequence of events which will have a high probability of influencing the patients' state in a desired manner.

Formally, the concept of state--changing sequential patterns can be defined as follows.
By a \textit{sequence} $s=\text{\textless} s_1, s_2, ..., s_n\text{\textgreater}$ we understand an ordered multi--set of \textit{elements}, where each element $s_i$ is drawn from the same set.
We distinguish two types of sequences: sequences of events $\mathcal{S}^e$ and sequences of states $\mathcal{S}^s$, where the elements in the events sequences are drawn from the events set and the elements in the states sequences are drawn from the ordered states set.
Each events sequence $s^e\in\mathcal{S}^e$ has a corresponding states sequence $s^{se}\in\mathcal{S}^s$ (note that this correspondence is only one--sided).
The corresponding sequences can be combined into a single sequence of events and states and there exists a total order between the elements of the combined sequences such that the order of the elements from each sequence is preserved.

A sequence $s'$ which elements form a subset of elements of another sequence $s$ is called a \textit{subsequence} of $s$ and is denoted as $s'\subseteq s$.
Given a set of sequences $\mathcal{S}$, a sequence $p$ is called a \textit{sequential pattern} (or \textit{pattern}), if it is a subsequence of at least \textit{minsup} sequences in $\mathcal{S}$: $|\{s\in\mathcal{S}:p\subseteq s\}|\geq\textit{minsup}$, where \textit{minsup} is a user--defined minimal support parameter.
We denote that a sequence $s$ \textit{contains} a pattern $p$ if $p\subseteq s$.
Given a sequence $s=\text{\textless} s_1,s_2,...,s_n\text{\textgreater}$, its subsequence $s'=\text{\textless} s_{i_1}, s_{i2}, ..., s_{i_m}\text{\textgreater}$, $1\leq m\leq n$, and an element $s_x\in s$, we say that $s_x$ appears in $s$ after $s'$ if $i_m<x\leq n$, and before $s'$ if $1\leq x<i_1$.
A \textit{state--changing sequential pattern} is a pattern which...

\section{Experiment}
\label{sec:experiments}


\section{Conclusions}
\label{sec:conclusions}
In this paper, ...

As the issue tackled in this paper is itself novel, the possibilities of extending this work are very broad.
As our most immediate future research focus, we plan on experimenting with introducing time constraints to the problem.
The constraints could concern both, events (e.g., restricting time gaps between events) and states (e.g., the certainty of a given object's state can decay over time until new state appears).
Moreover, we intend to create other sequential pattern evaluation measures dedicated for this specific problem.
We would also like to quantify the differences between states and include those differences in the analysis, as our current approach only registers the changes in states' direction.

\subsubsection*{Acknowledgments} This research is partly funded by the Polish National Science Center under Grant No. DEC-2015/19/B/ST6/02637.

\bibliographystyle{splncs}
\bibliography{Bibliography}

\end{document}
